\documentclass[12pt,letterpaper]{article}
\usepackage[utf8]{inputenc}
\usepackage[spanish,mexico]{babel}
\usepackage{amsmath}
\usepackage{amsfonts}
\usepackage{amssymb}
\usepackage{amsmath}
\usepackage[lmargin=3cm,rmargin=3cm,tmargin=3cm,bmargin=3cm]{geometry}

\usepackage{hyperref}
\usepackage{graphicx}
\usepackage{float}


\begin{document}

\title{Actividad 2: Primeros programas en Python}
\author{Luisa Fernanda Orci Fernandez.}
\date{30 de Enero del 2016}

\maketitle

\section*{Python}

Python es un lenguaje de programación interpretado, esto quiere decir que puede analizar y ejecutar otros programas. La diferencia entre un programa intérprete y un compilador, es que los compiladores traducen un programa desde un lenguaje de programación a código máquina y los interpretes como Python solo realizan la traducción conforme sea necesario. \\
Python es administrado por Python Software Foundation, posee una licencia de código abierto y es compatible con GNU. 

\section*{Actividad a realizar}

En esta ctividad se nos proporcionaron distintos programas en Python, los cuales hacían diferentes tipos de cálculos matemáticos, para que los modificaramos un poco y así nos familiarizaramos mas con lo que es programar en Python.


\section{Problema 1: Caída libre}

"Se deja caer una pelota desde el techo de una torre de altura $h$. Se desea saber la altura de la pelota respecto a la torre a un determinado tiempo después de haber sido dejada caer".

El programa proporcionado para calcular los resultados fue el siguiente: 

\begin{verbatim}
h = float(input("Proporciona la altura de la torre: "))
t = float(input("Ingresa el tiempo: "))
s = 0.5*9.81*t**2
print("La altura de la pelota es", h-s, "metros")
\end{verbatim}

A partir de este programa teníamos que crear uno que calculara el tiempo en que tarda la pelota en recorrer una distancia $h$ de la torre, despejamos el tiempo $t$ de $h-s$ y nos quedó de la siguiente forma:

$$ t = \sqrt{\frac{h}{(.5)(9.8)}}$$

Y el programa final fue el siguiente: 

\begin{verbatim}
from math import sqrt
h = float(input("Proporciona la altura de la torre: "))
t = sqrt(h/(.5*9.8))
print ("El tiempo que tarda la pelota en llegar al suelo es", t, "segundos")
\end{verbatim}

Ejemplo de los cálculos realizados por el programa:

\begin{verbatim}
Proporciona la altura de la torre: 20
('El tiempo que tarda la pelota en llegar al suelo es', 
2.0203050891044216, 'segundos')
\end{verbatim}
 
\section{Problema 2: Un sátelite orbitando alrededor de la Tierra}

" Un satélite orbita la Tierra a una altura $h$, con un periodo $T$ en segundos.
Demuestre que la altitud h del satélite sobre la superficie de la Tierra esta dado por la expresión:
 
\begin{equation}
(R + h^3) = \frac{(GMT^2}{4\pi^2}
\end{equation} 
 
donde: \\ $ G = 6.67x10^{-11} \frac{m^3}{kg(s)}$ es la constante de Gravitación Universal de Newton, $ M = 5.97x10^{24} kg $ es la masa de la Tierra y $ R = 6371000 m $ es su radio". 

A partir de este problema debíamos realizar un programa al cual le proporcionabamos el periodo $T$ en segundos y este nos calcularía la altura $h$ a la cual se encontraba el sátelite. 
A partir de la ecuación (1), despejamos a la altura $h$ y la ecuación resultante fue la siguiente: 

\begin{equation}
h = \sqrt[3]{\frac{(GMT^2}{4\pi^2}} - R
\end{equation}

Utiliamos esta ecuación en nuestro programa para que calculara el valor de la altura, el programa quedó así:

\begin{verbatim}

from math import pi
T = float(input("Ingrse un valor para el periodo en segundos" ))
G = 6.67e-11
M = 5.97e24
R = 6371000
h = (((G*M*T*T)/(4*pi*pi))**(1./3.)) - R
print ("El valor de la altitud del satelite sobre la superficie terrestre es", 
h, "metros")

\end{verbatim}

Para probar el programa ingresamos un periodo de 90 minutos, que corresponde a 5400 segundos y el resultado fue el siguiente: 

\begin{verbatim}
Ingrse un valor para el periodo en segundos5400
('El valor de la altitud del satelite sobre la superficie terrestre es',
 279321.6253728606, 'metros')
\end{verbatim}

\section{Problema 3: Coordenadas esféricas}
Estas coordenadas se basan en la misma idea que las polares, y son utilizadas frecuentemente para determinar la posición de un punto en el espacio, utilizando una distancia y dos ángulos. 
El punto $P$ se representa por tres magnitudes: el radio $r$, el ángulo polar $\phi$ y $\theta$.

El siguiente programa pide solicita un punto en coordenadas cartesianas y las convierte a coordenadas polares.

\begin{verbatim}
from math import sin, cos, pi, acos, atan, sqrt
x = float(input("Introduce un valor para x:"))
y = float(input("Introduce un valor para y:"))
z = float(input("Introduce un valor para z:"))

r = sqrt((x**2) + (y**2) + (z**2))
theta = acos((z/r))
phi = atan((y/x))

print("r = ", r, "theta = ", theta, "phi = ", phi)
\end{verbatim}

Corriendo el programa: 

\begin{verbatim}
Introduce un valor para x:5
Introduce un valor para y:10
Introduce un valor para z:5
('r = ', 12.24744871391589, 'theta = ', 1.1502619915109316,
 'phi = ', 1.1071487177940904)
\end{verbatim}

\section{Problema 4: Números pares e impares}
En este problema solo corrimos un programa que se nos proporcionó, el objetivo de esto es ver como funciona el comando $while$. El programa detecta los números pares e impares: 

\begin{verbatim}
print("Enter two integers, one even, one odd.")
m = int(input("Enter the first integer: "))
n = int(input("Enter the second integer: "))
while (m+n)%2==0:
    print("One must be even and the other odd.")
    m = int(input("Enter the first integer: "))
    n = int(input("Enter the second integer: "))
print("The numbers you chose are",m,"and",n)
\end{verbatim}

Con esto nos dimos cuenta que el comando $while$ sirve para hacer iteraciones.

\section{Problema 5: Números de Catalan}

Este problema consistía en crear un programa que calculara los números de Catalan menores o igual a $1000000$. El problema anterior fue de gran ayuda, ya que nos mostró como utilizar el $while$, también fue de gran ayuda el ejemplo de como crear un programa para calcular la sucesión de números de Fibonacci: 

\begin{verbatim}
f1, f2 = 1, 1
while f2<1000:
    print(f2)
    f1, f2 = f2, f1+f2
\end{verbatim}
también se nos proporcionó la fórmula de recurrencia de los números de Catalan: $$ C_0 =1, C_{n+1} = \frac{2)(2n+1)}{(n+2)} Cn  $$

El programa quedó de la siguiente manera:

\begin{verbatim}
n, c = 0., 1.
while c<=1000000:
    print(c)
    n, c = (n+1), (2*(2*n+1)/(n+2))*c
\end{verbatim}

Y el resultado fue:

\begin{verbatim}

1.0
1.0
2.0
5.0
14.0
42.0
132.0
429.0
1430.0
4862.0
16796.0
58786.0
208012.0
742900.0

\end{verbatim}

\section*{Conclusiones}

Me pareció que esta actividad será de gran ayuda en un futuro, ya que los programas que hicimos se ven sencillos, pero contienen comandos que van a ser de gran ayuda. 


\begin{thebibliography}{widestlabel}
\bibitem{w} Wikipedia, https://en.wikipedia.org
\bibitem{w} http://computacional1.pbworks.com/w/page/104476954/Actividad%202%20(2016-1)
\end{thebibliography}





\end{document}